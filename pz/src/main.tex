%% Преамбула TeX-файла

% 1. Стиль и язык
\documentclass[utf8x]{src/styles/G7-32} % Стиль (по умолчанию будет 14pt)
\usepackage[T2A]{fontenc}
\usepackage[russian]{babel}
% Остальные стандартные настройки убраны в preamble.inc.tex.
\include{configs/preamble.inc}

% Настройки листингов.
\include{configs/listings.inc}

% Полезные макросы листингов.
\include{configs/macros.inc}

\begin{document}

    \includepdf[pages=-]{titles/0-title.pdf}
    \includepdf[pages=-]{titles/10-task.pdf}
    \includepdf[pages=-]{titles/20-annotation.pdf}

    \setcounter{page}{4}

    \frontmatter % выключает нумерацию ВСЕГО; здесь начинаются ненумерованные главы: реферат, введение, глоссарий, сокращения и прочее.

% Команды \breakingbeforechapters и \nonbreakingbeforechapters
% управляют разрывом страницы перед главами.
% По-умолчанию страница разрывается.

% \nobreakingbeforechapters
% \breakingbeforechapters

    \tableofcontents

    \Abbreviations %% Список обозначений и сокращений в тексте
\begin{description}
    \item[ЭВМ] Электронно-вычислительная машина
    \item[ВС] Вычислительная система
    \item[ПО] Программное обеспечение
    \item[DevOps] Development & Operations, <<Разработка и Эксплуатация>>
    \item[SaaS] Software as a Service, <<Программное обеспечение как услуга>>
    \item[IaC] Infrastructure as Code, <<Инфраструктура как код>>
    \item[CI/CD] Continuous Integration/Continuous Deployment, <<Продолжительная интеграция/Продолжительная развёртка>>
    \item[VCS] Version Control System, <<Система контроля версий>>
\end{description}

%%% Local Variables:
%%% mode: latex
%%% TeX-master: "rpz"
%%% End:


    \Introduction

Актуальность темы.
Веб технологии широко распространились в нашем мире и на сегодняшний день почти каждая ВС взаимодействует со всемирной паутиной.
В свою очередь, поддержка и разработка наиболее популярной архитектуры <<Клиент-Сервер>> таких систем требует существенных временных затрат, поскольку уже с самого начала проектирования требуется решить ряд следующих задач:

\begin{itemize}
    \item вертикальное и горизонтальное масштабирование системы;
    \item доставка обновлений сервиса на рабочие ЭВМ;
    \item управление окружением ВС;
    \item осуществления контроля качества поступающих изменений;
    \item управление версиями ВС;
    \item бесшовное развёртывание отдельных компонентов системы;
    \item оперативная загрузка срочных исправлений.
\end{itemize}

В рамках названого проекта по автоматизации управлением жизненным циклом ВС, автором было предложено решить данные задачи.
Результаты выполнения задания должны быть представлены в выпускной квалификационной работе.

На основании выданного технического задания определяется цель исследования: разработать простое и доступное в установке и поддержке ПО для автоматизации управления жизненным циклом веб-сервиса.

Для реализации цели исследования ставятся следующие задачи исследования:

\begin{itemize}
    \item рассмотреть исходные условия для применения к автоматизации управления жизненным циклом ВС;
    \item рассмотреть преимущества и недостатки ПО применительно к задачам управления жизненным циклом;
    \item рассмотреть аппаратно-программные средства, которые могут быть использованы при выполнении работ по автоматизации управлением жизненным циклом ВС;
    \item провести практические работы по разработке простого и доступного в установке и поддержке ПО для автоматизации управления веб-сервиса;
    \item обосновать полученные результаты;
    \item разработать предложения по использованию полученных результатов в практической деятельности автоматизации управления веб-сервисом.
\end{itemize}

Поставленные задачи определяют предмет исследования:

Простое и доступное в установке и поддержке ПО для автоматизации управления жизненным циклом веб-сервиса.

Объект исследования: ПО, задействованные и которые могут быть использованы для автоматизации управления жизненным циклом веб-сервиса.

При проведении исследований были просмотрены 10 источников, в том числе 5 электронных ресурсов, что отражено в библиографическом списке.
При написании выпускной квалификационной работы (ВКР) из библиографического списка использовано 5 источников.

ВКР состоит из введения, основной части, включающей три главы, заключения и приложений.
В ВКР содержится 10 страниц основного текста, 0 рисунков, 0 таблиц и 0 приложений.

    \mainmatter % это включает нумерацию глав и секций в документе ниже

    \chapter{ОБЗОР СУЩЕСТВУЮЩИХ УЧЕБНЫХ СТЕНДОВ НА БАЗЕ БЕСКОЛЛЕКТОРНЫХ ДВИГАТЕЛЕЙ}
\label{cha:analysis}

\section{Преимущества и недостатки ПО применительно к задачам управления жизненным циклом}

\section{Применение GitLab и Kubernetes для автоматизации управления жизненным циклом}

\cite{kubertenes}
\cite{gitlab}

\section{Недостатки GitLab и Kubernetes}


%%% Local Variables:
%%% mode: latex
%%% TeX-master: "rpz"
%%% End:

    \chapter{ПРОЕКТИРОВАНИЕ СИСТЕМЫ АВТОМАТИЗАЦИИ УПРАВЛЕНИЯ ЖИЗНЕННЫМ ЦИКЛОМ ВЕБ-СЕРВИСА}
\label{cha:design}

\section{Анализ требований к системе}

Так как полученный для равёртки веб-сервисе базируется на Node.js и исходный код разбит на библиотеки, то то для корректной
работы системы потребуется наличие регистров Node пакетов.
Помимо регистра пакетов необходимым будет регистр Docker образов, поскольку веб-сервис уже представлен в виде Docker образов
и имеет все необходимые образы ПО необходимые окружению.

Согласно требованиям были сформулированы основные действующими лица (актёры) в работе системы:
\begin{itemize}
    \item администратор --- пользователь занимающийся настройкой прав доступа другим пользователям
        и конфигурацией развёртки веб-сервиса,
    \item разработчик --- пользователь системы имеющий доступ к репозиториям с исходным кодом,
        хранилищам пакетов и контейнеров, а так же управлению релизами веб-сервиса,
    \item сотрудник отдела качества (тестировщик) --- пользователь системы имеющий доступ к просмотру аналитеческих данных и
        проведению автоматизированных тестовых сценариев внутри заранее подготовленных окружений.
\end{itemize}

Пользователь системы подразумевает любого актёра.
На основании описания актёров и их основных возможностей была составлена диаграмма случаев использования.
Изображение данной диаграммы представлено на Рисунке \ref{fig:use-cases}.

\begin{figure}[h!]
    \centering
    \begin{tikzpicture}
        \begin{umlsystem}[x=5]{system}{Система}
            \umlusecase[width=3cm]{Управление правами доступа}
            \umlusecase[y=-4, width=3cm]{Управление развёрткой}
            \umlusecase[x=6, y=-2, width=3cm]{Конфигурация окружений}
            \umlusecase[x=6, y=-6, width=3cm]{Конфигурация системных ограничений}

            \umlusecase[y=-8, width=3cm]{Получение доступа к исходному коду и хранилищам}
            \umlusecase[y=-12, width=3cm]{Управление релизами сервиса и библиотек}

            \umlusecase[x=6, y=-14, width=3cm]{Проведение автоматизированного тестирования}
            \umlusecase[y=-16, width=3cm]{Сбор аналитических данных}
        \end{umlsystem}

        \umlactor[y=-2]{Администратор}{admin}
        \umlactor[y=-10]{Разработчик}{developer}
        \umlactor[y=-14]{Тестировщик}{qa}

        \umlassoc{admin}{usecase-1}
        \umlassoc{admin}{usecase-2}
        \umlassoc{developer}{usecase-5}
        \umlassoc{developer}{usecase-6}
        \umlassoc{qa}{usecase-7}
        \umlassoc{qa}{usecase-8}

        \umlinclude{usecase-3}{usecase-2}
        \umlinclude{usecase-4}{usecase-2}
    \end{tikzpicture}
    \caption{Диаграмма случаев использования системы}
    \label{fig:use-cases}
\end{figure}

Согласно требованиям системой должно поддерживаться три основных рабочих окружения веб-сервиса под различные цели:

\begin{itemize}
    \item develop --- инсценировка рабочего окружения веб-сервива для разработчиков,
    \item testing --- окружение для проведения ручного тестирования и сбора аналитических данных,
    \item release --- рабочее окружение веб-сервиса для реальных пользователей.
\end{itemize}

С точки зрения CI/CD взаимодействие пользователя с системой сосредоточено вокруг комита в репозиторий и автоматическим запуском
одной или нескольких задач (Jobs) внутри определённой линии (Pipeline).
Каждая задача является набором последовательно исполняемых инструкций ожидаемо завершённых без ошибок.
В случае ошибки выполнение всей линии завершается и повторяется только по действию пользователя.
При этом линии задач строятся динамически в зависимости от конкретного репозитория и ветки, поэтому некоторые задачи могут быть пропущены при выполнении,
а некоторые комиты оставаться без линий задач вовсе.
Результами работы линий являются артефакты, которые содержар основную информацию о результатах работы системы.
В общем виде поведение системы отражено на Рисунке \ref{fig:system-common}.

\begin{figure}[h!]
    \centering
    \begin{tikzpicture}
        \begin{umlseqdiag}
            \umlactor[class=А]{Пользователь}{a}
            \umldatabase[class=B]{Репозиторий}{b}
            \umlobject[class=C]{Pipeline}{c}
            \umlobject[class=D]{Job}{d}
            \begin{umlcall}[op=коммит, type=asynchron]{a}{b}
                \begin{umlcall}[op=запускает, type=asynchron]{b}{c}
                    \begin{umlcall}[op=выполняет, type=synchron, return=0]{c}{d}
                    \end{umlcall}
                    \begin{umlcall}[op=пропущена, type=synchron]{c}{d}
                    \end{umlcall}
                    \begin{umlcall}[op=выполняет, type=synchron, return=1]{c}{d}
                    \end{umlcall}
                    \begin{umlcall}[op=пропущена, type=synchron]{c}{d}
                    \end{umlcall}
                \end{umlcall}
            \end{umlcall}
            \begin{umlcall}[op=коммит, type=asynchron]{a}{b}
            \end{umlcall}
            \begin{umlcall}[op=коммит, type=asynchron]{a}{b}
            \end{umlcall}
            \begin{umlcall}[op=коммит, type=asynchron]{a}{b}
                \begin{umlcall}[op=запускает, type=asynchron, return=артефакты]{b}{c}
                    \begin{umlcall}[op=выполняет, type=synchron, return=0]{c}{d}
                    \end{umlcall}
                    \begin{umlcall}[op=пропущена, type=synchron]{c}{d}
                    \end{umlcall}
                    \begin{umlcall}[op=выполняет, type=synchron, return=0]{c}{d}
                    \end{umlcall}
                    \begin{umlcall}[op=выполняет, type=synchron, return=0]{c}{d}
                    \end{umlcall}
                \end{umlcall}
            \end{umlcall}
        \end{umlseqdiag}
    \end{tikzpicture}
    \caption{Общий случай работы системы}
    \label{fig:system-common}
\end{figure}

Так как линия задач зависит от репозитория, то необходимо систематизировать репозитории в системе:

\begin{itemize}
    \item репозиторий с исходным кодом компонента веб-сервиса --- может быть несколько, обязательно содержит Dockerfile в корне,
        предоставляется полный доступ разработчикам, доступ к аналитическим данным тестировщику.
    \item репозиторий с конфигурациями развёртки --- только один, содержит общие скрипты и конфигурации, предоставляется доступ только администратору,
    \item репозиторий с исходным кодом библиотек компонентов веб-сервиса --- только один, предоставляется полный доступ разработчикам.
    доступ к аналитическим данным тестировщику.
\end{itemize}

Таким образом, основными объектами в системе являются:
\begin{itemize}
    \item пользователь с набором прав доступа,
    \item репозиторий одного из типов,
    \item линия с задачами.
\end{itemize}

\section{Проектирование линий задач}

Для проектирования был отдельно разобран каждый случай использования и описан в виде линии задач:
\begin{itemize}
    \item проведение автоматизированного тестирования --- линия будет срабатывать на каждую синхронизацию с основной веткой удаленного репозитория и проводить разные виды тестирования,
    \item управление релизами сервиса и библиотек --- линия будет составляться в зависимости от ветки репозитория, собирать исходный код, загружать готовый к работе код в хранилище и
    оповещать кластер о выходе обновления при необходимости,
    \item управление развёрткой --- линия будет заключаться в применении обновлённых конфигураций окружения из репозитория к кластеру,
    \item получение доступа к исходному коду и хранилищам --- случай использования будет реализовываться не средствами CI/CD,
    \item сбор аналитических данных --- данные будут предоставляться артефактами в резульате работы линий задач,
    \item управление правами доступа --- случай использования будет реализовываться не средствами CI/CD.
\end{itemize}

Самым частым этапом является проведение автоматизированного тестирования.
Так как система заранее не может знать о возможных сценариях использования веб-сервиса, то вся ответственность о их содержании переносится на тестировщика.
В целом, процесс тестирования будет происходить в несколько основных этапов: семантическое тестирования исходного кода, юнит и интеграционное тестирование библиотек и
определённых сервисов компонентов системы.
На основании данных этапов была составлена линия задач процесса тестирования веб-сервиса:

\begin{itemize}
    \item lint --- семантическое тестирования исходного кода путём запуска встроенного скрипта модуля разработчиками,
    \item test --- юнит и интеграционное тестирование библиотек веб-сервиса путём запуска скрипта модуля разработчиками и предоставление артефактов выполнения,
\end{itemize}

Создание релиза будет происходить похоже на создание релиза в git flow, только к комитам привязаны действия линий задач:
исполнение коммита при помощи git, проведение тестирования, сборка образа и оповещение кластера.
На основании данных этапов была составлена линия задач создания релиза веб-сервиса:

\begin{itemize}
    \item test --- осуществление контроля качества путём запуска задач линии тестирования,
    \item build --- сборка исходного кода нужной версии и загрузкой в хранилище в зависимости от требуемого окружения (опциональная задача, требует подтверждения пользователем),
    \item publish --- оповещение кластера или зависимых сервисов о выходе обновления.
\end{itemize}

Данная линия будет иметь два аргумента: название сервиса компонента, которые необходимо обновить, и название окружения, в котором необходимо произвести релиз.
Так как шаги сборки исходного кода и оповещения о релизе зависят лишь от входных инструкций Dockerfile, то данные шаги будут
скрыты от разработчиков веб-сервиса в репозитории с конфигурациями развёртки.
Разработчику необходимо будет только импортировать необходимые задачи из репозитория.

Конфигурация развёртки веб-сервиса состоит из линии, включающей только одну задачу: применение конфигураций развёртки к кластеру.
В целях структуризации конфигураций для этих целей будет использоваться задача publish из линии по созданию релиза, которая при отсутствии аргументов будет обновлять конфигурации рахвёртки кластера.

\section{Выбор сервисов архитекутры}

Проводя обзор доступных на рынке git хостингов, можно сделать вывод, что наиболее распространенным git хостингом на сегодняшний день является хостинг компании GitLab Inc.
К тому же, по соотношению цена-функционал хостинг этой компании существенно обходит конкурентов.
Также, к существенному преимуществу можно отнести наличие обширного сообщества пользователей и разработчиков программных решений на основе git хостинга GitLab,
что позволяет иметь доступ к множеству готовых решений и получать помощь в разработке при необходимости.

Для реализации поставленной в данной работе задачи гибкость настройки всей инфраструктуры окружения не требуется, а также ставится в приоритет скорость ввода в рабочее состояние.
Поэтому в качестве оркестратора вместо гибкости Kubernetes был выбран Docker Swarm\cite{devOpsPhy}.
Но так как в данной работе делается акцент на гибкость всей системы, то далее будет рассмотрена описание конфигурации для использования с Kubernetes, не считая уставноку и настройку самого кластера.
Для работы будет подготовлена одна вершина Docker Swarm в статусе мэнеджер, поскольку более не требуется на данном этапе.

Одной и ключевой настройкой является открытие портов на уровне операционной системы сервера\cite{web:docker:swarm}:

\begin{itemize}
    \item TCP порт 2377 для коммуникации между менеджерами кластера,
    \item TCP и UDP порты 7946 для взаимодействия между нодами кластера,
    \item UDP порта 4789 для управления сетевым трафиком.
\end{itemize}

Как было сказано ранее, для развёртки в работе используется Docker, поэтому необходим простой инструмент удалённого доступа к сокету Docker сервиса на рабочем сервере.
Для этих целей будет использоваться GitLab Runner с установленным исполнителем задач Docker.
Кратко описать работу ранера можно следующим образом: раннер запускается в отдельном контейнере с добавленным volume на сокет Docker,
таким образом получается избежать достаточно сложного и нелесообразного запуска Docker внутри Docker,
так как в этом случае раннер получает доступ напрямую к сокету Docker сервера.

Данное решение имеет потенциальную проблему с безопасностью, поскольку если злоумышленник получит доступ к описанию задач GitLab CI/CD, то он сможет запускать на рабочем сервере любое ПО.
Для избежания данной проблемы будут установлены настройки доступа внутри GitLab.
Так же для избежания потери полезного времени работы ранера, необходимо будет произвести настройку кэша ранера.
Ключевой настройкой явлется политика загрузки образов для задач, поскольку по умолчанию ранер в любом случае будет загружать образ из регистра, даже если образ представлен локально.
Согласно требованиям ранер должен будет запускать минимум три задачи за единицу времени, данное значение будет отражено в конфигурации на этапе реализации.

\section{Составление плана тестирования}

В качестве метода тестирования было выбрано тестирования белого ящика, так как исходный код системы известен.

На основании диаграммы случаев использования были описаны следующие основные тестовые сценарии и представлены в виде таблицы \ref{tab:testing-plan}.

\begin{center}
    \begin{longtable}{|c|p{0.15\textwidth}|p{0.2\textwidth}|p{0.2\textwidth}|p{0.2\textwidth}|}
        \caption{План тестирования}
        \label{tab:testing-plan}
        \hline
        № & Название тестового сценария             & Тестовый сценарий                                                                                         & Тестовые данные                           & Ожидаемый результат \\
        \hline
        1 & Линия задач автоматизации тестироания  & Открыть репозитории сервисов, проверить срабатывание и составление линий автоматизированного тестирования & Список сервисов компонентов веб-сервиса   & Линии собираются правильно, задачи выполняются, артефакты предоставляются \\
        \hline
        2 & Линия задач релиза создания релиза  & Открыть репозитории сервисов, проверить срабатывание и составление линий создания релиза & Список сервисов компонентов веб-сервиса   & Линии собираются правильно, задачи выполняются, веб-сервиса обновляется \\
        \hline
        3 & Линия задач управления развёрткой  & Открыть репозиторий развёртки, изменить конфигурации развёртки, зафиксировать изменения коммитом & Конфигурации развёртки веб-сервиса   & Веб-сервис правильно реагирует на изменение конфигураций развёртки \\
        \hline
    \end{longtable}
\end{center}

%%% Local Variables:
%%% mode: latex
%%% TeX-master: "rpz"
%%% End:

    \chapter{РЕАЛИЗАЦИЯ СИСТЕМЫ АВТОМАТИЗАЦИИ УПРАВЛЕНИЯ ЖИЗНЕННЫМ ЦИКЛОМ ВЕБ-СЕРВИСА}
\label{cha:impl}

\section{Подготовка GitLab}

Перед реализацией необходимо произвести базовую настройку окружения.
Для этого был зарегистрирован GitLab аккаунт, установлен SSH ключ и создана группа проекта.
Результаты создания группы представлены на рисунке \ref{fig:group-ready}.

\begin{figure}[ht]
    \centering
    \includegraphics[scale=0.4]{src/figures/group-ready}
    \caption{Скриншот создания группы в GitLab}
    \label{fig:group-ready}
\end{figure}

Следующим этапом было произведено создание необходимых репозиториев с последующей загрузкой исходного кода предоставленного для работы веб-сервиса:

\begin{itemize}
    \item web-client --- репозиторий для хранения исходного кода веб-клиента проекта,
    \item api --- репозиторий для хранения исходного кода API проекта.
    \item node-packages --- репозиторий для хранения исходного кода общих npm зависимостей проекта.
\end{itemize}

Так же в корне каждого репозитория был загружен Dockerfile для развёртывания данного сервиса.
Так как в качестве модели ветвления была выбрана модель git flow, то так же были подготовлены соответствующие ветки в репозиториях под разные окружения: develop, testing и release.
Установка доступа к данным репозиториям предоставляется только разработчикам данных программных решений в целях безопасности.

Подготовка хранилищ npm пакетов и Docker образов не требуется, так как GitLab берёт на себя данную ответственность и не требует дополнительных действий от пользователя.

На следующем этапе был подготовлен репозиторий deployment для хранения общих скриптов и конфигураций окружения и развёртывания.
Установка доступа к этому репозиторию предоставляется только команде обеспечения развёртывания в целях безопасности.
Результаты создания репозиториев представлены на рисунке \ref{fig:reps-ready}.

\begin{figure}[ht]
    \centering
    \includegraphics[scale=0.4]{src/figures/reps-ready}
    \caption{Скриншот создания репозиториев в GitLab}
    \label{fig:reps-ready}
\end{figure}

\section{Установка кластера Docker Swarm}

Следующим этапом был установлен и настроен Docker Swarm кластер.

Для этого на рабочей системе было произведено открытие необходимых портов в операционной системе\cite{linuxPocket}:
\begin{lstlisting}[language=bash,caption={Открытие портов в Linux}]
$ sudo ufw allow 2377
$ sudo ufw allow 7946
$ sudo ufw allow 4789
\end{lstlisting}

Дальше была произведена инициализация кластера на сервере:
\begin{lstlisting}[language=bash,caption={Инициализация кластера}]
$ docker swarm init --advertise-addr 192.168.1.101
  Swarm initialized: current node (dxn1zf6l61qsb1josjja83ngz) is now a manager.
  To add a worker to this swarm, run the following command:
    docker swarm join --token <token> 192.168.1.101:2377
  To add a manager to this swarm, run 'docker swarm join-token manager' and follow the instructions.
\end{lstlisting}

В данном этапе нет необходимости при использовании Kubernetes.

\section{Установка и настройка GitLab Runner}

После была произведена установка и настройка GitLab Runner на рабочем сервере.
В качестве дистрибутива на этапе проектирования был выбран Docker, как самый быстрый и удобный.
После установки необходимо зарегистрировать GitLab Runner, для этого на странице настроек CI/CD группы в GitLab был получен регистрационный токен.
Конфигурация runner производится путём редактирования config.toml\cite{web:gitlab:docs} файла в соответствии с этапом проектирования, основными настройками являются:

\begin{itemize}
    \item Установка executor --- docker executor,
    \item Установка volumes --- /var/run/docker.sock\cite{web:docker:docs},
    \item Установка pull-policy --- if-not-present,
    \item Установка concurrent --- 3.
\end{itemize}

На данном этапе runner полностью готов к работе и ожидает входящих задач.

\section{Описание CI/CD конфигураций}

На следующем этапе необходимо подготовить общие для работы сервисов конфигурации запуска, которые будут храниться в репозитории deployment:

\begin{itemize}
    \item build.yaml --- набор универсальных задач, предназначенный для сборки Docker образа и последующей загрузки в регистр контейнеров,
    \item publish.yaml --- набор универсальных задач, предназначенный для оповещения кластера об обновлении сервиса для загрузки новой версии,
    \item publish-api.yaml --- расширенная версия publish.yaml, содержащая дополнительные задачи для проведения миграция базы данных в случае необходимости,
\end{itemize}

Одним из наиболее интересных файлов конфигураций является .gitlab-ci.yaml, поскольку в нём содержится основная логика управления развёрткой веб-сервиса.
На данных строчках содержится автоматическое подключение к кластеру Docker образов без с учётом риска передачи паролей через переменные окружения.
Так же в данных строчках содержится описание стадий задач, необходимых для развёртки сервиса api.
Пример приведён на листинге \ref{lst:api-tasks}.

\lstinputlisting[language=yaml,caption=Задачи сервиса api (\Code{deployment/.gitlab-ci.yml}),label=lst:api-tasks]{src/listings/api-tasks.yaml}

Данная задача используется для безопасного обновления конкретного сервиса по его названию и названию окружения, которая добавляется в линию только при указании значения названия сервиса и ветки.
Пример приведён на листинге \ref{lst:update-a-service}.

\lstinputlisting[language=yaml,caption=Обновление сервиса (\Code{deployment/.gitlab-ci.yml}),label=lst:update-a-service]{src/listings/update-a-service.yaml}

В конце файла содержится задача, условием запуска которой является отсутствие названия сервиса и веток.
Данная задача используется для экстренного перезапуска веб-сервиса.
Пример приведён на листинге \ref{lst:update-service}.

\lstinputlisting[language=yaml,caption=Перезапуск веб-сервиса (\Code{deployment/.gitlab-ci.yml}),label=lst:update-service]{src/listings/update-service.yaml}

Далее была произведена конфигурация на уровне сервисов компонентов, в репозитории api и web-client были добавлены конфигурации CI/CD путём создания .gitlab-ci.yml файла, содержащего основные задачи.

Для репозитория node-packages были описаны задачи с использованием CLI lerna, которая предназначена для организации работы моно-репозитория.
Использование этой библиотеки обусловлено большим сообществом разработчиков и открытостью исходного кода.
Данная программа обновляет версии библиотек при помощи построения и обхода графа зависимостей, построенного на основании package.json файлов.
Пример данного графа в условиях данного веб-сервиса приведён на рисунке \ref{fig:dependency-graph}.

\begin{figure}[ht]
    \centering
    \includegraphics[scale=0.4]{src/figures/dependency-graph}
    \caption{Граф зависимостей компонентов веб-сервиса}
    \label{fig:dependency-graph}
\end{figure}

\section{Развёртка сервисов внутри кластера}

Завершающим этапом реализации является описание конфигурационных файлов Docker Swarm.
Для этого в настройках CI/CD репозитория deployment были добавлены переменные окружения (секреты) на все рабочие окружения (develop, testing и release),
содержащие аргументы сервисов компонентов системы (доступы к базе данных, секрет ключа авторизации и так далее)\cite{kuberForDevOps}.
Аналогично были добавлены Docker Swarm конфигурационные файлы под каждый окружения:

\begin{itemize}
    \item develop --- каждый сервис запускается на сервере в одном экземпляре, ресурсы сервера сильно ограничены во избежание лишней нагрузки,
    \item testing --- аналогичен develop, только используется для целей ручного тестирования веб-сервиса,
    \item release --- web-client запускается в одном экземпляре, база данных и api запускаются в трёх, большая часть ресурса отведена под эти сервисы.
\end{itemize}

На листинге \ref{lst:api-services} приведён пример части конфигурационного файла Docker Swarm для запуска сервиса api.

\lstinputlisting[language=yaml,caption=Сервисы api (\Code{deployment/docker-compose.master.yaml}),label=lst:api-services]{src/listings/api-services.yaml}

На листинге \ref{lst:web-client-service} приведён пример части конфигурационного файла Docker Swarm для запуска сервиса web-client.

\lstinputlisting[language=yaml,caption=Сервисы web-client (\Code{deployment/docker-compose.master.yaml}),label=lst:web-client-services]{src/listings/web-client-services.yaml}

В итоге веб-сервис был развёрнут, пример открытия веб страницы приведён на рисунке \ref{fig:app-screen}.

\begin{figure}[ht]
    \centering
    \includegraphics[scale=0.4]{src/figures/app-screen}
    \caption{Скриншот открытия страницы веб-сервиса}
    \label{fig:app-screen}
\end{figure}

%%% Local Variables:
%%% mode: latex
%%% TeX-master: "rpz"
%%% End:

    \chapter{Экспериментальный раздел}
\label{cha:research}

В данном разделе проводятся вычислительные эксперименты.
А на рис.~\ref{fig:spire01} показана схема мыслительного процесса автора...

\begin{figure}
    \centering
    \includegraphics[scale=0.1]{figures/pic01}
    \caption{Как страшно жить}
    \label{fig:spire01}
\end{figure}


%%% Local Variables:
%%% mode: latex
%%% TeX-master: "rpz"
%%% End:


    \backmatter %% Здесь заканчивается нумерованная часть документа и начинаются ссылки и
    %% заключение

    \Conclusion % заключение к отчёту

Результаты выполнения задания на выпускную квалификационную работу, - позволяют сделать следующие выводы:
\begin{itemize}
    \item
    Цель исследования достигнута;
    полученный веб-сервис был развёрнут по наиболее популярным и современным методологиям IaC, CI/CD и SaaS.

    \item
    В рамках решения задачи: Провести обзор необходимых средств для автоматизации управления жизненным циклом веб-сервиса,
    - была выбрана VCS Git, было проведено сравнение git хостингов с анализом преимуществ и недостатков, а так же проведено сравнение оркестраторов контейнеров.

    \item
    В рамках решения задачи: Рассмотреть полученный для развёртки веб-сервис и проанализировать требования,
    - была установлена необходимость хранилищ пакетов и образов, а так же составлена диаграмма случаев использования системы.

    \item
    В рамках решения задачи: Провести проектирование механизмов автоматизации управления жизненного цикла веб-сервиса,
    - были описаны и представлены в виде линий задач случаи использования системы.

    \item
    В рамках решения задачи: Составить план тестирования механизмов развёртки веб-сервиса,
    - были сформулированы цели и задачи тестирования, была выбрана методология тестирования, а так же были описаны тестовые сценарии.

    \item
    В рамках решения задачи: Провести практические работы по развёртке и автоматизации управления жизненного цикла веб-сервиса,
    - был подготовлен для работы GitLab вместе с окружением в виде GitLab Runner и Docker Swarm, а так же развёрнут веб-сервис.

    \item
    В рамках решения задачи: Провести тестирование механизмов развёртки и обосновать полученные результаты,
    - было произведено ручное тестирование по методологии <<<Белого ящика>> функций полученной системы..
\end{itemize}


%%% Local Variables: 
%%% mode: latex
%%% TeX-master: "rpz"
%%% End: 

    % % Список литературы при помощи BibTeX
% Юзать так:
%
% pdflatex rpz
% bibtex rpz
% pdflatex rpz

\bibliographystyle{src/styles/gost780u}
\bibliography{src/rpz}

%%% Local Variables:
%%% mode: latex
%%% TeX-master: "rpz"
%%% End:


    \appendix   % Тут идут приложения
%
%    \include{90-appendix1}
%    \include{91-appendix2}

\end{document}

%%% Local Variables:
%%% mode: latex
%%% TeX-master: t
%%% End:
