\chapter{Конструкторский раздел}
\label{cha:design}

В данном разделе проектируется новая всячина.

\section{Архитектура всячины}

\paragraph{Проверка} параграфа. Вроде работает.
\paragraph{Вторая проверка} параграфа. Опять работает.

Вот.

\begin{itemize}
    \item Это список с <<палочками>>.
    \item Хотя он и не по ГОСТ, кажется.
\end{itemize}

\begin{enumerate}
    \item Поэтому для списка, начинающегося с заглавной буквы, лучше список с цифрами.
\end{enumerate}

Формула \ref{F:F1} совершено бессмысленна.

%Кстати, при каких-то условиях <<удавалось>> получить двойный скобки вокруг номеров формул. Вопрос исследуется.

\begin{equation}
    a= cb
    \label{F:F1}
\end{equation}


Окружение \texttt{cases} опять работает (см. \ref{F:F2}), спасибо И. Короткову за исправления..


\begin{equation}
    a= \begin{cases}
           3x + 5y + z, \mbox{если хорошо} \\
           7x - 2y + 4z, \mbox{если плохо}\\
           -6x + 3y + 2z, \mbox{если совсем плохо}
    \end{cases}
    \label{F:F2}
\end{equation}

\section{Подсистема всякой ерунды}

Культурная вставка dot-файлов через утилиту dot2tex (рис.~\ref{fig:fig02}).

\begin{figure}
% [width=0.5\textwidth] --- регулировка ширины картинки
    \includegraphics{figures/pic01}
    \caption{Рисунок}
    \label{fig:fig02}
\end{figure}

\begin{figure}
    \centering
    \begin{tikzpicture}
        \begin{umlsystem}[x=4]{system}{Автоматизация жизненным циклом веб-сервиса}
            \umlusecase[width=3cm]{Управление правами доступа}
            \umlusecase[y=-4, width=3cm]{Управление развёрткой}
            \umlusecase[x=6, width=3cm]{Получить актуальную версию веб-сервиса}
            \umlusecase[x=6, y=-4, width=3cm]{Получить тестовую версию веб-сервиса}
        \end{umlsystem}

        \umlactor{Разработчик}{developer}
        \umlactor[y=-3]{Админнистратор}{admin}
        \umlactor[x=14]{Пользователь}{user}
        \umlactor[x=14,y=-3]{Сотр. отдела качества}{qa}

%        \umlassoc{user}{usecase-1}

%        \umlinherit{subuser}{user}
%        \umlassoc{user}{usecase-1}
%        \umlassoc{subuser}{usecase-2}
%        \umlassoc{subuser}{usecase-3}
%        \umlinherit{usecase-2}{usecase-1}
%        \umlVHextend{usecase-5}{usecase-4}
%        \umlinclude[name=incl]{usecase-3}{usecase-4}
    \end{tikzpicture}
    \caption{UML диаграмма}
    \label{fig:fig03}
\end{figure}


\subsection{Блок-схема всякой ерунды}

\subsubsection*{Кстати о заголовках}

У нас есть и \Code{subsubsection}. Только лучше её не нумеровать.

Проводя обзор доступных на рынке git хостингов, можно сделать вывод, что наиболее распространенным git хостингом на сегодняшний день является хостинг компании GitLab Inc.
К тому же, по соотношению цена-функционал хостинг этой компании существенно обходит конкурентов.
Также, к существенному преимуществу можно отнести наличие обширного сообщества пользователей и разработчиков программных решений на основе git хостинга GitLab,
что позволяет иметь доступ к множеству готовых решений и получать помощь в разработке при необходимости.

Для реализации поставленной в данной работе задачи гибкость настройки всей инфраструктуры окружения не требуется, а также ставится в приоритет скорость ввода в рабочее состояние.
Поэтому в качестве оркестратора вместо гибкости Kubernetes был выбран Docker Swarm.
Но так как в данной работе делаеся акцент на гибкость всей системы, то далее будет рассмотрена описание конфигурации для использования с Kubernetes, не считая уставноку и настройку самого кластера.
Для работы будет подготовлена одна вершина Docker Swarm в статусе мэнеджер, поскольку более не требуется на данном этапе.

Как было сказано ранее, для развёртки в работе используется Docker, поэтому необходим простой инструмент удлаенного доступа к сокету Docker сервиса на рабочем сервере.
Для этих целей будет использоваться GitLab Runner с установленным исполнителем задач Docker.
Кратко описать работу ранера можно следующим образом: раннер запускается в отдельном контейнере с добавленным volume на сокет Docker,
таким образом получается избежать достаточно сложного и нелесообразного запуска Docker внутри Docker,
так как в этом случае раннер получает доступ напрямую к сокету Docker сервера.
Данное решение имеет потенциальную проблему с безопасностью, поскольку если злоумышленник получит доступ к описанию задач GitLab CI/CD, то он сможет запускать на рабочем сервере любое ПО.
Для избежания данной проблемы будут установлены настройки доступа внутри GitLab.
Так же для избежания потери полезного времени работы ранера, необходимо будет произвести настройку кэша ранера.
Ключевой настройкой явлется политика загрузки образов для задач, поскольку по умолчанию ранер в любом случае будет загружать образ из регистра, даже если образ представлен локально.
Согласно требованиям ранер должен будет запускать минимум три задачи за единицу времени, данное значение будет отражено в конфигурации на этапе реализации.

Автоматизация контроля качества будет реализована путём запуска описанных разработчиками тестовых сценариев внутри Pipelines.
Отчёты по прохождению сценариев будут представлены в веб интерфейсе GitLab путём интеграции с CI/CD.

Для контроля версий будет использована модель ветвления git flow со следующими окружениями:

\begin{itemize}
    \item develop --- инсценировка рабочего окружения веб серсива для разработчиков,
    \item testing --- окружение для проведения ручного тестирования,
    \item release --- рабочее окружение сервиса для реальных пользователей.
\end{itemize}

Основная проблема заключается в обновлении пакетов зависимостей проекта, поскольку для обновления необходимо пройти засимимым сервисам и построить граф.
Для решения этой проблемы будет разработано программное решение на базе CI/CD c применением bash скриптов при помощи package.json файлов веб сервиса.

Для проведения тесирования был составлен план содержащий набор ключевых функция для тестирвоания.
Подробнее план представлен в виде таблице с описанием сценариев, входных данных и ожидаемого результата в тестке работы ВКР.


%%% Local Variables:
%%% mode: latex
%%% TeX-master: "rpz"
%%% End:
