\chapter{ПРОЕКТИРОВАНИЕ СИСТЕМЫ АВТОМАТИЗАЦИИ УПРАВЛЕНИЯ ЖИЗНЕННЫМ ЦИКЛОМ ВЕБ-СЕРВИСА}
\label{cha:design}

\section{Анализ требований к системе}

Так как полученный для равёртки веб-сервисе базируется на Node.js и исходный код разбит на библиотеки, то то для корректной
работы системы потребуется наличие регистров Node пакетов.
Помимо регистра пакетов необходимым будет регистр Docker образов, поскольку веб-сервис уже представлен в виде Docker образов
и имеет все необходимые образы ПО необходимые окружению.

Согласно требованиям были сформулированы основные действующими лица (актёры) в работе системы:
\begin{itemize}
    \item администратор --- пользователь занимающийся настройкой прав доступа другим пользователям
        и конфигурацией развёртки веб-сервиса,
    \item разработчик --- пользователь системы имеющий доступ к репозиториям с исходным кодом,
        хранилищам пакетов и контейнеров, а так же управлению релизами веб-сервиса,
    \item сотрудник отдела качества (тестировщик) --- пользователь системы имеющий доступ к просмотру аналитеческих данных и
        проведению автоматизированных тестовых сценариев внутри заранее подготовленных окружений.
\end{itemize}

Пользователь системы подразумевает любого актёра.
На основании описания актёров и их основных возможностей была составлена диаграмма случаев использования.
Изображение данной диаграммы представлено на Рисунке \ref{fig:use-cases}.

\begin{figure}[h!]
    \centering
    \begin{tikzpicture}
        \begin{umlsystem}[x=5]{system}{Система}
            \umlusecase[width=3cm]{Управление правами доступа}
            \umlusecase[y=-4, width=3cm]{Управление развёрткой}
            \umlusecase[x=6, y=-2, width=3cm]{Конфигурация окружений}
            \umlusecase[x=6, y=-6, width=3cm]{Конфигурация системных ограничений}

            \umlusecase[y=-8, width=3cm]{Получение доступа к исходному коду и хранилищам}
            \umlusecase[y=-12, width=3cm]{Управление релизами сервиса и библиотек}

            \umlusecase[x=6, y=-14, width=3cm]{Проведение автоматизированного тестирования}
            \umlusecase[y=-16, width=3cm]{Сбор аналитических данных}
        \end{umlsystem}

        \umlactor[y=-2]{Администратор}{admin}
        \umlactor[y=-10]{Разработчик}{developer}
        \umlactor[y=-14]{Тестировщик}{qa}

        \umlassoc{admin}{usecase-1}
        \umlassoc{admin}{usecase-2}
        \umlassoc{developer}{usecase-5}
        \umlassoc{developer}{usecase-6}
        \umlassoc{qa}{usecase-7}
        \umlassoc{qa}{usecase-8}

        \umlinclude{usecase-3}{usecase-2}
        \umlinclude{usecase-4}{usecase-2}
    \end{tikzpicture}
    \caption{Диаграмма случаев использования системы}
    \label{fig:use-cases}
\end{figure}

Согласно требованиям системой должно поддерживаться три основных рабочих окружения веб-сервиса под различные цели:

\begin{itemize}
    \item develop --- инсценировка рабочего окружения веб-сервива для разработчиков,
    \item testing --- окружение для проведения ручного тестирования и сбора аналитических данных,
    \item release --- рабочее окружение веб-сервиса для реальных пользователей.
\end{itemize}

С точки зрения CI/CD взаимодействие пользователя с системой сосредоточено вокруг комита в репозиторий и автоматическим запуском
одной или нескольких задач (Jobs) внутри определённой линии (Pipeline).
Каждая задача является набором последовательно исполняемых инструкций ожидаемо завершённых без ошибок.
В случае ошибки выполнение всей линии завершается и повторяется только по действию пользователя.
При этом линии задач строятся динамически в зависимости от конкретного репозитория и ветки, поэтому некоторые задачи могут быть пропущены при выполнении,
а некоторые комиты оставаться без линий задач вовсе.
Результами работы линий являются артефакты, которые содержар основную информацию о результатах работы системы.
В общем виде поведение системы отражено на Рисунке \ref{fig:system-common}.

\begin{figure}[h!]
    \centering
    \begin{tikzpicture}
        \begin{umlseqdiag}
            \umlactor[class=А]{Пользователь}{a}
            \umldatabase[class=B]{Репозиторий}{b}
            \umlobject[class=C]{Pipeline}{c}
            \umlobject[class=D]{Job}{d}
            \begin{umlcall}[op=коммит, type=asynchron]{a}{b}
                \begin{umlcall}[op=запускает, type=asynchron]{b}{c}
                    \begin{umlcall}[op=выполняет, type=synchron, return=0]{c}{d}
                    \end{umlcall}
                    \begin{umlcall}[op=пропущена, type=synchron]{c}{d}
                    \end{umlcall}
                    \begin{umlcall}[op=выполняет, type=synchron, return=1]{c}{d}
                    \end{umlcall}
                    \begin{umlcall}[op=пропущена, type=synchron]{c}{d}
                    \end{umlcall}
                \end{umlcall}
            \end{umlcall}
            \begin{umlcall}[op=коммит, type=asynchron]{a}{b}
            \end{umlcall}
            \begin{umlcall}[op=коммит, type=asynchron]{a}{b}
            \end{umlcall}
            \begin{umlcall}[op=коммит, type=asynchron]{a}{b}
                \begin{umlcall}[op=запускает, type=asynchron, return=артефакты]{b}{c}
                    \begin{umlcall}[op=выполняет, type=synchron, return=0]{c}{d}
                    \end{umlcall}
                    \begin{umlcall}[op=пропущена, type=synchron]{c}{d}
                    \end{umlcall}
                    \begin{umlcall}[op=выполняет, type=synchron, return=0]{c}{d}
                    \end{umlcall}
                    \begin{umlcall}[op=выполняет, type=synchron, return=0]{c}{d}
                    \end{umlcall}
                \end{umlcall}
            \end{umlcall}
        \end{umlseqdiag}
    \end{tikzpicture}
    \caption{Общий случай работы системы}
    \label{fig:system-common}
\end{figure}

Так как линия задач зависит от репозитория, то необходимо систематизировать репозитории в системе:

\begin{itemize}
    \item репозиторий с исходным кодом компонента веб-сервиса --- может быть несколько, обязательно содержит Dockerfile в корне,
        предоставляется полный доступ разработчикам, доступ к аналитическим данным тестировщику.
    \item репозиторий с конфигурациями развёртки --- только один, содержит общие скрипты и конфигурации, предоставляется доступ только администратору,
    \item репозиторий с исходным кодом библиотек компонентов веб-сервиса --- только один, предоставляется полный доступ разработчикам.
    доступ к аналитическим данным тестировщику.
\end{itemize}

Таким образом, основными объектами в системе являются:
\begin{itemize}
    \item пользователь с набором прав доступа,
    \item репозиторий одного из типов,
    \item линия с задачами.
\end{itemize}

\section{Проектирование линий задач}

Для проектирования был отдельно разобран каждый случай использования и описан в виде линии задач:
\begin{itemize}
    \item проведение автоматизированного тестирования --- линия будет срабатывать на каждую синхронизацию с основной веткой удаленного репозитория и проводить разные виды тестирования,
    \item управление релизами сервиса и библиотек --- линия будет составляться в зависимости от ветки репозитория, собирать исходный код, загружать готовый к работе код в хранилище и
    оповещать кластер о выходе обновления при необходимости,
    \item управление развёрткой --- линия будет заключаться в применении обновлённых конфигураций окружения из репозитория к кластеру,
    \item получение доступа к исходному коду и хранилищам --- случай использования будет реализовываться не средствами CI/CD,
    \item сбор аналитических данных --- данные будут предоставляться артефактами в резульате работы линий задач,
    \item управление правами доступа --- случай использования будет реализовываться не средствами CI/CD.
\end{itemize}

Самым частым этапом является проведение автоматизированного тестирования.
Так как система заранее не может знать о возможных сценариях использования веб-сервиса, то вся ответственность о их содержании переносится на тестировщика.
В целом, процесс тестирования будет происходить в несколько основных этапов: семантическое тестирования исходного кода, юнит и интеграционное тестирование библиотек и
определённых сервисов компонентов системы.
На основании данных этапов была составлена линия задач процесса тестирования веб-сервиса:

\begin{itemize}
    \item lint --- семантическое тестирования исходного кода путём запуска встроенного скрипта модуля разработчиками,
    \item test --- юнит и интеграционное тестирование библиотек веб-сервиса путём запуска скрипта модуля разработчиками и предоставление артефактов выполнения,
\end{itemize}

Создание релиза будет происходить похоже на создание релиза в git flow, только к комитам привязаны действия линий задач:
исполнение коммита при помощи git, проведение тестирования, сборка образа и оповещение кластера.
На основании данных этапов была составлена линия задач создания релиза веб-сервиса:

\begin{itemize}
    \item test --- осуществление контроля качества путём запуска задач линии тестирования,
    \item build --- сборка исходного кода нужной версии и загрузкой в хранилище в зависимости от требуемого окружения (опциональная задача, требует подтверждения пользователем),
    \item publish --- оповещение кластера или зависимых сервисов о выходе обновления.
\end{itemize}

Данная линия будет иметь два аргумента: название сервиса компонента, которые необходимо обновить, и название окружения, в котором необходимо произвести релиз.
Так как шаги сборки исходного кода и оповещения о релизе зависят лишь от входных инструкций Dockerfile, то данные шаги будут
скрыты от разработчиков веб-сервиса в репозитории с конфигурациями развёртки.
Разработчику необходимо будет только импортировать необходимые задачи из репозитория.

Конфигурация развёртки веб-сервиса состоит из линии, включающей только одну задачу: применение конфигураций развёртки к кластеру.
В целях структуризации конфигураций для этих целей будет использоваться задача publish из линии по созданию релиза, которая при отсутствии аргументов будет обновлять конфигурации рахвёртки кластера.

\section{Выбор сервисов архитекутры}

Проводя обзор доступных на рынке git хостингов, можно сделать вывод, что наиболее распространенным git хостингом на сегодняшний день является хостинг компании GitLab Inc.
К тому же, по соотношению цена-функционал хостинг этой компании существенно обходит конкурентов.
Также, к существенному преимуществу можно отнести наличие обширного сообщества пользователей и разработчиков программных решений на основе git хостинга GitLab,
что позволяет иметь доступ к множеству готовых решений и получать помощь в разработке при необходимости.

Для реализации поставленной в данной работе задачи гибкость настройки всей инфраструктуры окружения не требуется, а также ставится в приоритет скорость ввода в рабочее состояние.
Поэтому в качестве оркестратора вместо гибкости Kubernetes был выбран Docker Swarm\cite{devOpsPhy}.
Но так как в данной работе делается акцент на гибкость всей системы, то далее будет рассмотрена описание конфигурации для использования с Kubernetes, не считая уставноку и настройку самого кластера.
Для работы будет подготовлена одна вершина Docker Swarm в статусе мэнеджер, поскольку более не требуется на данном этапе.

Одной и ключевой настройкой является открытие портов на уровне операционной системы сервера\cite{web:docker:swarm}:

\begin{itemize}
    \item TCP порт 2377 для коммуникации между менеджерами кластера,
    \item TCP и UDP порты 7946 для взаимодействия между нодами кластера,
    \item UDP порта 4789 для управления сетевым трафиком.
\end{itemize}

Как было сказано ранее, для развёртки в работе используется Docker, поэтому необходим простой инструмент удалённого доступа к сокету Docker сервиса на рабочем сервере.
Для этих целей будет использоваться GitLab Runner с установленным исполнителем задач Docker.
Кратко описать работу ранера можно следующим образом: раннер запускается в отдельном контейнере с добавленным volume на сокет Docker,
таким образом получается избежать достаточно сложного и нелесообразного запуска Docker внутри Docker,
так как в этом случае раннер получает доступ напрямую к сокету Docker сервера.

Данное решение имеет потенциальную проблему с безопасностью, поскольку если злоумышленник получит доступ к описанию задач GitLab CI/CD, то он сможет запускать на рабочем сервере любое ПО.
Для избежания данной проблемы будут установлены настройки доступа внутри GitLab.
Так же для избежания потери полезного времени работы ранера, необходимо будет произвести настройку кэша ранера.
Ключевой настройкой явлется политика загрузки образов для задач, поскольку по умолчанию ранер в любом случае будет загружать образ из регистра, даже если образ представлен локально.
Согласно требованиям ранер должен будет запускать минимум три задачи за единицу времени, данное значение будет отражено в конфигурации на этапе реализации.

\section{Составление плана тестирования}

В качестве метода тестирования было выбрано тестирования белого ящика, так как исходный код системы известен.

На основании диаграммы случаев использования были описаны следующие основные тестовые сценарии и представлены в виде таблицы \ref{tab:testing-plan}.

\begin{center}
    \begin{longtable}{|c|p{0.15\textwidth}|p{0.2\textwidth}|p{0.2\textwidth}|p{0.2\textwidth}|}
        \caption{План тестирования}
        \label{tab:testing-plan}
        \hline
        № & Название тестового сценария             & Тестовый сценарий                                                                                         & Тестовые данные                           & Ожидаемый результат \\
        \hline
        1 & Линия задач автоматизации тестироания  & Открыть репозитории сервисов, проверить срабатывание и составление линий автоматизированного тестирования & Список сервисов компонентов веб-сервиса   & Линии собираются правильно, задачи выполняются, артефакты предоставляются \\
        \hline
        2 & Линия задач релиза создания релиза  & Открыть репозитории сервисов, проверить срабатывание и составление линий создания релиза & Список сервисов компонентов веб-сервиса   & Линии собираются правильно, задачи выполняются, веб-сервиса обновляется \\
        \hline
        3 & Линия задач управления развёрткой  & Открыть репозиторий развёртки, изменить конфигурации развёртки, зафиксировать изменения коммитом & Конфигурации развёртки веб-сервиса   & Веб-сервис правильно реагирует на изменение конфигураций развёртки \\
        \hline
    \end{longtable}
\end{center}

%%% Local Variables:
%%% mode: latex
%%% TeX-master: "rpz"
%%% End:
