\Introduction

Актуальность темы.
Веб технологии широко распространились в нашем мире и на сегодняшний день почти каждая ВС взаимодействует со всемирной паутиной.
В свою очередь, поддержка и разработка наиболее популярной архитектуры <<Клиент-Сервер>> таких систем требует существенных временных затрат, поскольку уже с самого начала проектирования требуется решить ряд следующих задач:

\begin{itemize}
    \item вертикальное и горизонтальное масштабирование системы;
    \item доставка обновлений сервиса на рабочие ЭВМ;
    \item управление окружением ВС;
    \item осуществления контроля качества поступающих изменений;
    \item управление версиями ВС;
    \item бесшовное развёртывание отдельных компонентов системы;
    \item оперативная загрузка срочных исправлений.
\end{itemize}

В рамках названого проекта по автоматизации управлением жизненным циклом ВС, автором было предложено решить данные задачи.
Результаты выполнения задания должны быть представлены в выпускной квалификационной работе.

На основании выданного технического задания определяется цель исследования: разработать простое и доступное в установке и поддержке ПО для автоматизации управления жизненным циклом веб-сервиса.

Для реализации цели исследования ставятся следующие задачи исследования:

\begin{itemize}
    \item рассмотреть исходные условия для применения к автоматизации управления жизненным циклом ВС;
    \item рассмотреть преимущества и недостатки ПО применительно к задачам управления жизненным циклом;
    \item рассмотреть аппаратно-программные средства, которые могут быть использованы при выполнении работ по автоматизации управлением жизненным циклом ВС;
    \item провести практические работы по разработке простого и доступного в установке и поддержке ПО для автоматизации управления веб-сервиса;
    \item обосновать полученные результаты;
    \item разработать предложения по использованию полученных результатов в практической деятельности автоматизации управления веб-сервисом.
\end{itemize}

Поставленные задачи определяют предмет исследования:

Простое и доступное в установке и поддержке ПО для автоматизации управления жизненным циклом веб-сервиса.

Объект исследования: ПО, задействованные и которые могут быть использованы для автоматизации управления жизненным циклом веб-сервиса.

При проведении исследований были просмотрены 10 источников, в том числе 5 электронных ресурсов, что отражено в библиографическом списке.
При написании выпускной квалификационной работы (ВКР) из библиографического списка использовано 5 источников.

ВКР состоит из введения, основной части, включающей три главы, заключения и приложений.
В ВКР содержится 10 страниц основного текста, 0 рисунков, 0 таблиц и 0 приложений.